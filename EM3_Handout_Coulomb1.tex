\documentclass{tufte-handout}
\usepackage{amsmath}
\usepackage{mathtools}
\usepackage{bm}
\usepackage{amsmath}
\usepackage{siunitx}
\sisetup{detect-all}
\usepackage{svg}
\usepackage[utf8x]{inputenc}
%\usepackage[greek,english]{babel} 
\usepackage{textcomp}
\usepackage{textgreek}
%\numberwithin{equation}{section}

\newcommand{\uvec}[1]{{\bm{\hat{\textnormal{\bfseries #1}}}}}
\newcommand{\ux}{\uvec{x}}
\newcommand{\uy}{\uvec{y}}
\newcommand{\uv}{\uvec{v}}
\newcommand{\vv}{\vec{v}}
\newcommand{\ua}{\uvec{a}}
\DeclarePairedDelimiter\abs{\lvert}{\rvert}%

\makeatletter
\providecommand\add@text{}
\newcommand\tagaddtext[1]{%
  \gdef\add@text{#1\gdef\add@text{}}}% 
\renewcommand\tagform@[1]{%
  \maketag@@@{\llap{\add@text\quad}(\ignorespaces#1\unskip\@@italiccorr)}%
}
\makeatother

\title{EM3 Worked Examples on Coulomb - Part 1}
\author{Professor Timothy Drysdale}
\begin{document}
\maketitle

\section{Unit vector}

Assume we are given an arbitrary  vector of non-unitary magnitude $\vec{v}$ in 2-D 
\begin{marginfigure}
\includesvg{unit_vector}
\end{marginfigure}
\begin{equation}
\label{eq:vector}
\vv = 3\ux + 2\uy, 
\end{equation}
which we wish to convert into a unit vector $\uv$. We must scale $\vv$ so its magnitude is unitary, which we can do by dividing by the magnitude $\abs{\vv}$
\begin{equation}
\label{eq:unitvector}
\uv = \frac{\vv}{\abs{\vv}}. 
\end{equation}
First we work out the magnitude of $\vv$
\begin{equation}
\label{eq:magnitude}
\abs{\vv} = \sqrt{3^2 + 2^2} = \sqrt{13},
\end{equation}
then substitute Eq.~\ref{eq:vector} and Eq.~\ref{eq:magnitude} into Eq.~\ref{eq:unitvector}, giving
\begin{equation}
\uv = \frac{3}{\sqrt{13}}\ux + \frac{2}{\sqrt{13}}\uy. 
\end{equation}

We can check the unit vector is the correct length by calculating its magnitude (which should be $1$):
\begin{align}
\abs{\uv} & = \sqrt{\left(\frac{3}{\sqrt{13}}\right)^2 + \left(\frac{2}{\sqrt{13}}\right)^2} \\
& =  \sqrt{\frac{3^2 + 2^2}{13}} \\
& = \sqrt{\frac{13}{13}} \\
& = 1
\end{align}

\section{Two charges acting on a third charge}

\emph{Given two charges of 2C, one each located at $(-1m,0)$ and $(0,-1m)$, what is the force on a test charge of 1C located at the origin $(0,0)$?}
\begin{marginfigure}
\includesvg{two_charges}
\end{marginfigure}
\vspace{0.5cm}

Assuming the force $F$ in Newtons exerted on one charge $Q_0$ by another $Q_1$ is directed in direction $\uvec{a}$, then the force can be calculated by
\begin{equation}
\vec{F} = \frac{Q_0Q_1\ua}{4\pi\epsilon{}r^2}\tagaddtext{[\si{\newton}].}
\end{equation}

For the case of the first charge at $(-1m,0)$, $r = 1m$, $Q_0 = 2C$, and $Q_1 = 1C$, giving
\begin{align}
\vec{F_x} & = \frac{2C.1C\ux}{4\pi\epsilon{}1^2} \\
     & = \frac{\ux}{2\pi\epsilon{}} \\
    & = 1.8\times10^{10}\ux\tagaddtext{[\si{\newton}].}
\end{align}

By symmetry, because the charges are the same distance from the test charge, 
\begin{equation}
\vec{F_y} = 1.8\times10^{10}\uy\tagaddtext{[\si{\newton}].}
\end{equation}

Thus the total force from our two charges is 
\begin{equation}
\vec{F} = 1.8\times10^{10}\ux + 1.8\times10^{10}\uy\tagaddtext{[\si{\newton}],}
\end{equation}
with a magnitude of 
\begin{align}
\abs{\vec{F}} & = \sqrt{\vec{F_x}^2+\vec{F_y}^2} \\
& = 2.54\times10^{10}\tagaddtext{[\si{\newton}].}
\end{align}

\end{document}
